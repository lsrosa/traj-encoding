\section{Multi-Agents}\label{sec: multi-agents}

\subsection{1x1}

\subsection{1xN}

\subsection{nxN}

\rev{just for testing} \cite{zhou2022hivt} In this paper, the encoding will be done in multiple steps. 
\begin{itemize}
	\item Extract for each agent its surrounding region (other agents and lanes) and encoded the corresponding past observations, using a coordinate system that is normalized to this agent last observed position and velocity, where we embed the agents own trajectory as well as the surrounding agents trajectory.
	\item We apply multi head cross attention with the agent as the query and its neighbors as key and values, and a recursive layer afterwards.
	\item The spatial encodings are then encoded temporally using a time axis transformer block
	\item Similar to the encoding of the neighbors, the encoded agent position is enhanced by cross attention with the neighboring lane segments (here, each vector in a polyline is treated individually)
	\item An transformer is used to encode the interactions with each agent, where their states are expanded by the translations and rotations between the coordinate systems.
	\item The decoder used is simple, however, the joint prediction are not really joint, just parallel
	
\end{itemize}
Here, using for the local encoding two different transformers, first along the agent axis and then along the time axis, significantly reduces the computational cost of the model.

\subsection{NxN}