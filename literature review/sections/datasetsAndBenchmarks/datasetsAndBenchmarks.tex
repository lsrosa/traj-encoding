\section{Datasets and Benchmarks}\label{sec: datasets and benchmarks}

\subsection{Datasets}

From \cite{xu2021tra2tra}: 
\begin{description}
	\item[41 ETH:] 2 pedestrians walking
	\item[42 UCY:] 2 pedestrian walking
	\item[43 Collisions:] Many simulations of balls rolling on a 2D plane without friction 
	\item[44 NGSim:] Freeway data with vehicles.
	\item[45 Charges:] Scenes of electric charges moving on an electric field
\end{description}

From \cite{ivanovic2019trajectron}:
\begin{description}
	\item[41 ETH:] 2 pedestrians walking
	\item[42 UCY:] 2 pedestrian walking
\end{description}

From \cite{lee2017desire}:
\begin{description}
	\item[12 KITTI Raw Data:] scenes of driving scenes with 3D velodyne laser scans
	\item[36 Stanford Drone Dataset] videos captures with a drone flying over an university
\end{description}


\method{2016 - Learning Social Etiquette: Human Trajectory Understanding In Crowded Scenes}

\cite{robicquet2016learning} provide a benchmark for social navigation with interactions between pedestrians, skaters, bikers, and small vehicles. 
%
The benchmark contain images (videos) of these interactions in a campus.

A feature called social sensitivity is proposed, which incorporates a distance which the target prefers for avoiding collision and another in which the targets starts to deviate from its trajectory to avoid collision.
%
These parameters are learned with an energy-like minimization.
%
When plotting these parameters, clusters emerge with different navigation styles.

\subsection{Simulators}

From \cite{okal2016learning}:
\begin{description}
	\item[22 ] open source pedestrian simulator governed by social force model
\end{description}

From \cite{kiss2021probabilistic}:
\begin{description}
	\item[2 Pedsim] pedestrian simulator used for generating dense pedestrian maps
\end{description}

\subsection{Benchmarks}

