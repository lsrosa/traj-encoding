\section{Social Robots}\label{sec: social robots}

\subsection*{2016 - Learning Social Etiquette: Human Trajectory Understanding In Crowded Scenes}

\cite{robicquet2016learning} provide a benchmark for social navigation with interactions between pedestrians, skaters, bikers, and small vehicles. 
%
The benchmark contain images (videos) of these interactions in a campus.

A feature called social sensitivity is proposed, which incorporates a distance which the target prefers for avoiding collision and another in which the targets starts to deviate from its trajectory to avoid collision.
%
These parameters are learned with an energy-like minimization.
%
When plotting these parameters, clusters emerge with different navigation styles.

\textbf{Notes:} No encoding here. 

\subsection*{2018 - Social GAN: Socially Acceptable Trajectories With Generative Adversarial Networks}

\textbf{SGAN}

\cite{gupta2018social} proposed an \gls{lstm} encoder-decoder with max-pooling layer for handling interdependencies (the trajectory of a person depends on the trajectory of others).
%
The encoder decoder is trained in a \gls{gan} fashion, in which uses an \gls{lstm}-based discriminator.

There is one \gls{lstm} encoder, decoder and discriminator for each tracked individual.
%
On the encoder, first the individual's position is passed through an $1$ layer \gls{mlp} to be transformed into a fixed-sized vector, which is then passed to the \gls{lstm}. The weights of the encoders are shared among all individuals.
%
The pooling module converts the hidden-state of every encoding \gls{lstm} into tensor for each individual.
%
The decoder is a straight-forward \gls{lstm} and the discriminator takes the predicted trajectories and classifies encodes them using \gls{lstms} into "good" or "bad".

The pooling mechanism is what handles multiple people.

\subsection*{2018 - Soft + Hardwired attention: An LSTM framework for human trajectory prediction and abnormal event detection}

\cite{fernando2018soft+}

\subsection*{2019 - Pedestrian Trajectory Prediction Using RNN Encoder-Decoder with Spatio-Temporal Attentions}

\cite{bhujel2019pedestrian}

\subsection*{2019 - GD-GAN: Generative Adversarial Networks for Trajectory Prediction and Group Detection in Crowds}

\cite{fernando2019gd}

\subsection*{2020 - Trajectron++: Dynamically-Feasible Trajectory Forecasting with Heterogeneous Data}

\cite{salzmann2020trajectron++}

\subsection*{2020 - It Is Not the Journey But the Destination: Endpoint Conditioned Trajectory Prediction}

\cite{mangalam2020not}

\subsection*{2020 - A Generative Approach for Socially Compliant Navigation}

\cite{tsai2020generative}

\subsection*{2021 - Learning World Transition Model for Socially Aware Robot Navigation}

\cite{cui2021learning}

\subsection*{2021 - Probabilistic Dynamic Crowd Prediction for Social Navigation}

\cite{kiss2021probabilistic}

\subsection*{2021 - Tra2Tra: Trajectory-to-Trajectory Prediction With a Global Social Spatial-Temporal Attentive Neural Network}

\cite{xu2021tra2tra}

\subsection*{2021 - Human Trajectory Forecasting in Crowds: A Deep Learning Perspective}

\cite{kothari2021human}

\subsection*{2021 - Trajectory Prediction for Autonomous Driving based on Multi-Head Attention with Joint Agent-Map Representation}

\cite{messaoud2021trajectory}

\subsection*{2022 - Social-PatteRNN: Socially-Aware Trajectory Prediction Guided by Motion Patterns}

\cite{navarro2022social}

\subsection*{2023 - CSR: Cascade Conditional Variational Auto Encoder with Socially-aware Regression for Pedestrian Trajectory Prediction}

\cite{zhou2023csr}

\subsection*{2023 - MRGTraj: A Novel Non-Autoregressive Approach for Human Trajectory Prediction}

\cite{peng2023mrgtraj}

\subsection*{2023 - EWareNet: Emotion-Aware Pedestrian Intent Prediction and Adaptive Spatial Profile Fusion for Social Robot Navigation}

\cite{narayanan2023ewarenet}
