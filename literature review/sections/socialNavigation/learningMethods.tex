\subsection{Learning Methods for Social Navigation}\label{subsec: learning social navigation}

\method{2016 - Learning Social Etiquette: Human Trajectory Understanding In Crowded Scenes}

\cite{robicquet2016learning} provide a benchmark for social navigation with interactions between pedestrians, skaters, bikers, and small vehicles. 
%
The benchmark contain images (videos) of these interactions in a campus.

A feature called social sensitivity is proposed, which incorporates a distance which the target prefers for avoiding collision and another in which the targets starts to deviate from its trajectory to avoid collision.
%
These parameters are learned with an energy-like minimization.
%
When plotting these parameters, clusters emerge with different navigation styles.

\textbf{Notes:} No encoding here. 

\method{2017 - DESIRE: Distant Future Prediction in Dynamic Scenes With Interacting Agents}

\textbf{Name: DESIRE}

\cite{lee2017desire}

\method{2018 - Soft + Hardwired attention: An LSTM framework for human trajectory prediction and abnormal event detection}

\cite{fernando2018soft+} proposes a framework which uses soft and hardwired attention mechanisms to predict human trajectories based on a brief history of the target's and its neighbors trajectories.

Trajectories for each pedestrian are encoded and decoded using \glspl{lstm} encoders and passes to the soft and hard attention layers to compute the final encoding, which is then decoded.
%
The key idea is to use the distance to compute weights for the hard attention layer, since these are key features that influence in the trajectory.

The approach is also evaluated in computing abnormal trajectories based on the proposed encoding.

\method{2019 - Pedestrian Trajectory Prediction Using RNN Encoder-Decoder with Spatio-Temporal Attentions}

\cite{bhujel2019pedestrian} proposes using not only the humans trajectories, but also information from the scene for predicting trajectories, using an \gls{rnn} for learning human-human and human-scene interactions, using attention mechanisms to find semantic alignment between the encoder and decoder.

Images as processed with a pre-trained \gls{cnn} for extracting environment features and attention mechanisms are computed before the decoding part.

\rev{this paper is very bad and should probably be kicked out}

\method{2019 - GD-GAN: Generative Adversarial Networks for Trajectory Prediction and Group Detection in Crowds}

\cite{fernando2019gd} proposes a framework which predicts trajectories and group memberships through clustering.
%
It builds on top of \cite{fernando2018soft+}, which uses \glspl{lstm} for learning an embedding for trajectories with attention mechanisms, here, the decoder is a \gls{gan} architecture with generators and discriminators being \glspl{lstm}.
%
The embedding is passed through a \gls{tsne} module for dimensionality reduction and a further clustering predicts group membership. 

\method{2019 - The Trajectron: Probabilistic Multi-Agent Trajectory Modeling With Dynamic Spatiotemporal Graphs}

\cite{ivanovic2019trajectron} presents the \texttt{trajectron} ...

\method{2020 - Trajectron++: Dynamically-Feasible Trajectory Forecasting with Heterogeneous Data}

\cite{salzmann2020trajectron++}  presents a graph structured recurrent model for learning trajectories considering dynamics and environment constraints (maps) by extending \cite{ivanovic2019trajectron} by adding support to multi-agents and heterogeneous data. The robot in question is an autonomous car traveling in a city.

The scene is represented as a graph, with nodes representing agents (cars and ppl), and edges representing interactions.
%
The scene evolution is encoded by a \gls{lstm} and attention is used to balance the weights in the interactions.
%
Finally a \gls{cnn} is used to aggregate heterogeneous data from a map, and semantic semantic information (pedestrian crossing", "drivable area", "walkway"). 
%
Multi-modality is achieved through the use of \gls{cvae} \rev{but it is not very clear where it is used}.

\method{2020 - It Is Not the Journey But the Destination: Endpoint Conditioned Trajectory Prediction}

\textbf{Name: PECNet}

\cite{mangalam2020not} focus on learning long-range multi-modal trajectory prediction, they propose a social-pooling layer which allows for improving the diversity of the predicted social-compliant trajectories.

Sets of trajectories and destination points are embedded, and the embeddings are uses by a \gls{vae} for computing predicted destination encoding, which is then passed to the social pooling for estimating which is the probable future destination.
%
Finally, the future destination encoding is used to estimate the trajectory.
%
\rev{the paper is quite confusing}

Tests are made with datasets of humans walking around.

\method{2020 - A Generative Approach for Socially Compliant Navigation}

\cite{tsai2020generative}

\method{2021 - Learning World Transition Model for Socially Aware Robot Navigation}

\cite{cui2021learning}

\method{2021 - Probabilistic Dynamic Crowd Prediction for Social Navigation}

\cite{kiss2021probabilistic}

\method{2021 - Tra2Tra: Trajectory-to-Trajectory Prediction With a Global Social Spatial-Temporal Attentive Neural Network}

\cite{xu2021tra2tra}

\method{2021 - Human Trajectory Forecasting in Crowds: A Deep Learning Perspective}

\cite{kothari2021human}

\method{2021 - Trajectory Prediction for Autonomous Driving based on Multi-Head Attention with Joint Agent-Map Representation}

\cite{messaoud2021trajectory}

\method{2022 - Social-PatteRNN: Socially-Aware Trajectory Prediction Guided by Motion Patterns}

\cite{navarro2022social}

\method{2023 - CSR: Cascade Conditional Variational Auto Encoder with Socially-aware Regression for Pedestrian Trajectory Prediction}

\cite{zhou2023csr}

\method{2023 - MRGTraj: A Novel Non-Autoregressive Approach for Human Trajectory Prediction}

\cite{peng2023mrgtraj}

\method{2023 - EWareNet: Emotion-Aware Pedestrian Intent Prediction and Adaptive Spatial Profile Fusion for Social Robot Navigation}

\cite{narayanan2023ewarenet}