\subsection{Pedestrian Prediction}\label{subsec: pedestrian prediction}

\method{2018 - Soft + Hardwired attention: An LSTM framework for human trajectory prediction and abnormal event detection}

\cite{fernando2018soft+} proposes a framework which uses soft and hardwired attention mechanisms to predict human trajectories based on a brief history of the target's and its neighbors trajectories.

Trajectories for each pedestrian are encoded and decoded using \glspl{lstm} encoders and passes to the soft and hard attention layers to compute the final encoding, which is then decoded.
%
The key idea is to use the distance to compute weights for the hard attention layer, since these are key features that influence in the trajectory.

The approach is also evaluated in computing abnormal trajectories based on the proposed encoding.

\method{2019 - Pedestrian Trajectory Prediction Using RNN Encoder-Decoder with Spatio-Temporal Attentions}

\cite{bhujel2019pedestrian} proposes using not only the humans trajectories, but also information from the scene for predicting trajectories, using an \gls{rnn} for learning human-human and human-scene interactions, using attention mechanisms to find semantic alignment between the encoder and decoder.

Images as processed with a pre-trained \gls{cnn} for extracting environment features and attention mechanisms are computed before the decoding part.

\rev{this paper is very bad and should probably be kicked out}

\method{2019 - GD-GAN: Generative Adversarial Networks for Trajectory Prediction and Group Detection in Crowds}

\cite{fernando2019gd} proposes a framework which predicts trajectories and group memberships through clustering.
%
It builds on top of \cite{fernando2018soft+}, which uses \glspl{lstm} for learning an embedding for trajectories with attention mechanisms, here, the decoder is a \gls{gan} architecture with generators and discriminators being \glspl{lstm}.
%
The embedding is passed through a \gls{tsne} module for dimensionality reduction and a further clustering predicts group membership. 

\method{2020 - It Is Not the Journey But the Destination: Endpoint Conditioned Trajectory Prediction}

\textbf{Name: PECNet}

\cite{mangalam2020not} focus on learning long-range multi-modal trajectory prediction, they propose a social-pooling layer which allows for improving the diversity of the predicted social-compliant trajectories.

Sets of trajectories and destination points are embedded, and the embeddings are uses by a \gls{vae} for computing predicted destination encoding, which is then passed to the social pooling for estimating which is the probable future destination.
%
Finally, the future destination encoding is used to estimate the trajectory.
%
\rev{the paper is quite confusing}

Tests are made with datasets of humans walking around.

\method{2022 - Social-PatteRNN: Socially-Aware Trajectory Prediction Guided by Motion Patterns}

\cite{navarro2022social}

\method{2023 - CSR: Cascade Conditional Variational Auto Encoder with Socially-aware Regression for Pedestrian Trajectory Prediction}

\cite{zhou2023csr}

\method{2023 - MRGTraj: A Novel Non-Autoregressive Approach for Human Trajectory Prediction}

\cite{peng2023mrgtraj}