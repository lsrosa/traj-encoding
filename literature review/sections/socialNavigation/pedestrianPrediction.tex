\subsection{Pedestrian Prediction}\label{subsec: pedestrian prediction}

\method{2018 - Soft + Hardwired attention: An LSTM framework for human trajectory prediction and abnormal event detection}

\cite{fernando2018soft+} proposes a framework which uses soft and hardwired attention mechanisms to predict human trajectories based on a brief history of the target's and its neighbors trajectories.

Trajectories for each pedestrian are encoded and decoded using \glspl{lstm} encoders and passes to the soft and hard attention layers to compute the final encoding, which is then decoded.
%
The key idea is to use the distance to compute weights for the hard attention layer, since these are key features that influence in the trajectory.

The approach is also evaluated in computing abnormal trajectories based on the proposed encoding.

\method{2019 - Pedestrian Trajectory Prediction Using RNN Encoder-Decoder with Spatio-Temporal Attentions}

\cite{bhujel2019pedestrian} proposes using not only the humans trajectories, but also information from the scene for predicting trajectories, using an \gls{rnn} for learning human-human and human-scene interactions, using attention mechanisms to find semantic alignment between the encoder and decoder.

Images as processed with a pre-trained \gls{cnn} for extracting environment features and attention mechanisms are computed before the decoding part.

\rev{this paper is very bad and should probably be kicked out}

\method{2019 - GD-GAN: Generative Adversarial Networks for Trajectory Prediction and Group Detection in Crowds}

\cite{fernando2019gd} proposes a framework which predicts trajectories and group memberships through clustering.
%
It builds on top of \cite{fernando2018soft+}, which uses \glspl{lstm} for learning an embedding for trajectories with attention mechanisms, here, the decoder is a \gls{gan} architecture with generators and discriminators being \glspl{lstm}.
%
The embedding is passed through a \gls{tsne} module for dimensionality reduction and a further clustering predicts group membership. 

\method{2020 - It Is Not the Journey But the Destination: Endpoint Conditioned Trajectory Prediction}

\textbf{Name: PECNet}

\method{2019 - The Trajectron: Probabilistic Multi-Agent Trajectory Modeling With Dynamic Spatiotemporal Graphs}

\rev{add to table}

\cite{ivanovic2019trajectron} presents the \texttt{trajectron}, a graph-structured model for predicting many (multi-modal) trajectories of multiple agents simultaneously (NxN).

They say their model is the first one to use graph representation, which has advantages over \glspl{rnn} since graphs naturally accept an arbitrary number of inputs \rev{as many agents as you want}, provide an intermediate representation and encourage model reuse.

First, past trajectories are encoded with \glspl{lstm}, which is then passed together with the \gls{lstm}-based ``edge-encoders'' for computing the influence of neighboring nodes and an additive attention module. 
%
The encoded values are passed to a \gls{cvae} and the decoder is also \gls{lstm}-based with a \gls{gmm} output which is sampled for generating multiple possible trajectories (achieving multi-modality).

They also propose a \gls{kde}-based evaluation metric, which is a bit more fair since multi-modal methods in the literature were generating thousands of trajectories and counting as a success if one of those hit the target.

Tested on datasets of ppl walking around.

\cite{mangalam2020not} focus on learning long-range multi-modal trajectory prediction, they propose a social-pooling layer which allows for improving the diversity of the predicted social-compliant trajectories.

Sets of trajectories and destination points are embedded, and the embeddings are uses by a \gls{vae} for computing predicted destination encoding, which is then passed to the social pooling for estimating which is the probable future destination.
%
Finally, the future destination encoding is used to estimate the trajectory.
%
\rev{the paper is quite confusing}

Tests are made with datasets of humans walking around.

\method{2021 - Human Trajectory Forecasting in Crowds: A Deep Learning Perspective}

\textbf{Name: DirectContact, TrajNet++}


\method{2021 - Tra2Tra: Trajectory-to-Trajectory Prediction With a Global Social Spatial-Temporal Attentive Neural Network}

\rev{add to the table}

\cite{xu2021tra2tra} proposes a global social spatial-temporal attentive \gls{nn}. 
%
It separates spatial interaction features with a ?decentralization operation? and attention mechanism, which is then passed to a \gls{lstm}, which is then passed together with velocities an auto-encoder-decoder.
%
Multi-modality is achieved by adding a random noise before decoding.

The decentralization subtracts the average positions from the trajectory.
%
The attention is a standard attention which gives weights for the features.
%
The \gls{lstm} are also standard, and their output is concatenated with velocity information.

tested on datasets with ppl walking around.

\cite{kothari2021human} presents a review on deep learning for human trajectory prediction, presents two methods for capturing social interaction and present TrajNet++, a benchmark for evaluating trajectory predictions. \rev{quite a lot of things in one paper}

They focus on short-term trajectory prediction (5 sec), since the long term objective cannot be observed.
%
They focus on the interactions of trajectory predictors, not on the predictor (\gls{lstm}) itself.
%
Specifically, to handle a variable number of neighbors and how they collective influence one's trajectory.

A pipeline is composed of an encoder, followed by a social model, and then a decoder.
%
Social model are classified in grid-based and non-grid-based, they propose a grid based method, using the velocities as observations, as it is natural for learning collision avoidance and leader-follower relations.
%
For grid-less method, they propose the \gls{lrp} which traces back which trajectories generate the prediction, and hence improving explainability.

\rev{This paper should also appear on the survey sections since it has a really good and complete literature review and comparison.}

\method{2022 - Social-PatteRNN: Socially-Aware Trajectory Prediction Guided by Motion Patterns}

\cite{navarro2022social}

\method{2023 - CSR: Cascade Conditional Variational Auto Encoder with Socially-aware Regression for Pedestrian Trajectory Prediction}

\cite{zhou2023csr}

\method{2023 - MRGTraj: A Novel Non-Autoregressive Approach for Human Trajectory Prediction}

\cite{peng2023mrgtraj}